%%This is a very basic article template.
%%There is just one section and two subsections.
\documentclass[14pt, a4paper]{article}
\usepackage[utf8]{inputenc}
\usepackage[english, russian]{babel}
\usepackage{tempora}
\usepackage[none]{hyphenat}
\usepackage{indentfirst}
\usepackage{ragged2e}
\usepackage{blindtext}

\begin{document}


%% Исследование и разработка методов построения инструментальных средств конфигурирования в плагинных системах

\section*{Системы информационно-управляющей среды ЛА}

\subsection*{Исследование принципов построения конфигуратора ARINC 653 спецификации в плагинных системах}

Разработка инструментальных средств конфигурирования для создания бортовой системы управления ЛА является важной проблемой в современной авионике. Настройка правил работы и поведения системы управления ЛА является трудоемкой задачей. Так, описание работы системы в соответствии со стандартом ARINC 653 включает в себя множество этапов, например определение перечня используемых в системе разделов, составление расписания выделения вычислительных мощностей системы с течением времени, настройка правил обработки ошибок и т.д. Разные аспекты ARINC конфигурирования требуют работы различных специалистов, которые работают в различных предметных областях.

Анализ показал, что применение унифицированного и немасштабирумого инструментального средства без возможности определения комплектации функционала, входящего в его состав, приведет к снижению эффективности работы профильных специалистов, потребует большего времени на изучение правил использования инструментального средства и снизит эффективность получаемой обратной связи от потребителя.

С целью разработки инструментальных средств с сохранением функционала вступать друг с другом в симбиотическую связь и возможностью автономной работы применяется технология плагинных систем. Важным аспектом в системах такого класса является управление зависимостями. Функционал компонентов должен быть распределен между друг другом так, чтобы объем функционала обязательный для осуществления поставки был минимальным. Математически такая задача решается поиском оптимальной декомпозиции. Ее решение должно обеспечивать возможность формирования максимального количества поставляемого функционала при минимальном количестве используемых плагинов. Для решения задачи необходимо описать предметную область и составить математическую модель проектируемой системы.

Процесс описания математической модели производился посредством исследования предметной области плагинных систем, выявления состава компонентов плагинной системы, анализа их связей друг с другом и ограничений на компоненты каждого типа.

В процессе исследования предметной области компоненты плагинной системы были классифицированы. Компоненты принадлежащие к одному классу могут быть связаны друг с другом, в том числе и циклично. Классы системы имеют иерархическую структуру и их иерархическое место строго детерминировано. Класс более высокой иерархической ступени не может существовать без дочерних элементов, описанных в виде классов меньшей ступени иерархии равно как и элементы классов нижних ступеней иерархии не могут быть описаны в системе без отношения к элементу более высокого класса. При возникновении связи между элементами принадлежащими различным элементам из класса более высокого уровня образуется связь между соответствующими элементами более высокого уровня иерархии.

Построение двумерной схемы на плоскости в виде графа с применением описанных правил на данных полученных при анализе инструментального средства конфигурирования для выполнения требований стандарта ARINC 653 дало схему распределения и графическое представление о наличии зависимостей между компонентами исследуемого средства. Эта информация позволила определить минимальный объем функционала, необходимый для формирования комплекта поставки инструментального средства.

Поиск оптимальной декомпозиции компонентов исследуемой системы производился при помощи алгоритма PageRank. Применение этого алгоритма позволило выявить наиболее загруженные элементы инструментального средства, в которых поток управления находился наибольшее количество времени при различных сценариях работы и выполнении различных пользовательских сценариев относящихся идеологически к различному функционалу исследуемого инструментального средства. Исходя из этого было выдвинуто предположение, что разделение функционала наиболее загруженных элементов системы, а так же исключение их одновременного вхождения в один плагин позволит сократить минимально требуемый функционал для формирования комплекта.

После формирования такого комплекта на поставку было проведено сравнение двух объемов поставляемого в комплекте функционала при одинаковом заявленном его составе. Результат показал уменьшение незаявленного функционала на 10 процентов.

Дальнейшая работа по исследованию методов построения инструментальных средств конфигурирования в плагинных системах предполагает выявление паттернов при проектировании инструментальных средств, исследование ограничений на систему в целом, а не на составные ее части, и анализ их влияния при составлении графовой математической модели, а так же применение других алгоритмов при анализе эффективности решения задачи оптимальной декомпозиции, в том числе алгоритмов машинного обучения.

\end{document}
